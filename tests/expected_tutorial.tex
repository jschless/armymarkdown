\documentclass{/app/resources/latex/armymemo-notikz}
\address{1st Training Battalion (Example)}
\address{1234 Army Drive}
\address{Fort Liberty, NC 28310}
\author{Sarah M. Johnson}\rank{CPT}\branch{MI}
\officesymbol{ATZB-CD-E}
\signaturedate{02 September 2025}
\subject{Army Markdown Tutorial and Formatting Guide}
\memoline{{MEMORANDUM FOR RECORD}}
\title{Company Commander}
\begin{document}
\begin{enumerate}
\item \textbf{PURPOSE}: This memorandum serves as a comprehensive tutorial for using Army Markdown to create professional military correspondence that is always perfectly formatted in accordance with AR 25-50.
\item \textbf{BACKGROUND}: Traditional word processors often result in formatting inconsistencies and errors in military memos. Army Markdown eliminates these issues by using simple text formatting that automatically generates properly formatted memos.
\item \textbf{BASIC FORMATTING RULES}:
\begin{enumerate}
\item Begin each main point with a dash (-) followed by a space
\item Sub-points are created by indenting 4 spaces (or one tab) before the dash
\item Further sub-points can be created with additional 4-space indentations
\item The system automatically handles numbering (1., 2., 3.) and lettering ((a), (b), (c))
\end{enumerate}
\item \textbf{TEXT FORMATTING OPTIONS}:
\begin{enumerate}
\item \textbf{Bold text}: Surround text with double asterisks: \textbf{this will be bold}
\item \textit{Italic text}: Surround text with single asterisks: \textit{this will be italicized}
\item \underline{Underlined text}: Surround text with triple asterisks: \underline{this will be underlined}
\item You can combine formatting: \underline{Bold and underlined} or \textbf{bold with \textit{italics} inside}
\end{enumerate}
\item \textbf{PARAGRAPH CONTINUATION}: You can add additional paragraphs to the same bullet point by leaving a blank line and continuing without a new dash, just like this paragraph demonstrates.
\item \textbf{DETAILED EXAMPLE OF NESTED STRUCTURE}:
\begin{enumerate}
\item This is a first-level sub-point (will show as "a.")
\item This is another first-level sub-point (will show as "b.")
\begin{enumerate}
\item This is a second-level sub-point (will show as "(1)")
\item Another second-level sub-point (will show as "(2)")
\begin{enumerate}
\item This is a third-level sub-point (will show as "(a)")
\item Final third-level sub-point (will show as "(b)")
\end{enumerate}
\end{enumerate}
\item Back to first-level (will show as "c.")
\end{enumerate}
\item \textbf{COMMON ARMY MEMO ELEMENTS}:
\begin{enumerate}
\item \textbf{Recommendations}: Use clear, actionable language
\item \textbf{Timelines}: Include specific dates and suspense requirements
\item \textbf{References}: Cite applicable regulations, policies, or directives
\item \textbf{Coordination}: Identify who needs to be informed or take action
\end{enumerate}
\item \textbf{BEST PRACTICES}:
\begin{enumerate}
\item Keep sentences concise and direct
\item Use active voice whenever possible
\item Avoid unnecessary jargon or acronyms without definition
\item Ensure each point flows logically to the next
\item Always include point of contact information
\end{enumerate}
\item \textbf{POINT OF CONTACT}: The point of contact for this memorandum is the undersigned at sarah.m.johnson@army.mil or (910) 555-0123.
\end{enumerate}
\end{document}
