\documentclass{/app/resources/latex/armymemo-notikz}
\address{ORGANIZATIONAL NAME/TITLE}
\address{STANDARDIZED STREET ADDRESS}
\address{CITY STATE 12345-1234}
\author{NAME (ALL CAPS)}\rank{Major}\branch{JA}
\officesymbol{OFFICE SYMBOL (ARIMS Record Number)}
\signaturedate{02 September 2025}
\subject{Preparing a Two-page Memorandum With a Suspense Date}
\memoline{MEMORANDUM FOR Joint Readiness Training Center, U.S. Army Forces Command, 6661 Warrior Trail Ave, Fort Polk, LA 71459}
\addencl{Personnel Listing, 22 March 2019}
\addencl{DA Form 4187}
\addencl{Orders 114-6}
\addencl{Locator}
\title{Chief, Claims Services}
\suspensedate{20 July 2023}
\begin{document}
\begin{enumerate}
\item Review this example to see how to prepare a memorandum. Allow 1 inch for the left, right and bottom margins.
\begin{enumerate}
\item Type the OFFICE SYMBOL at the left margin, two lines below the seal.
\item Stamp or type the DATE on the same line as the office symbol, flush to the right margin after signature. For instructions on how to place a text box for the application of dates to .pdf files with digital signature, see Appendix F. If there is a SUSPENSE DATE, type it two lines above the office symbol line flush to the right margin.
\item Type MEMORANDUM FOR on the third line below the office symbol. Begin the single address one space following MEMORANDUM FOR. If the MEMORANDUM FOR address extends more than one line, begin the second line flush with the left margin. Addresses may be in uppercase and lowercase type or all uppercase type. See the other figures within this chapter for preparing multiple-address memorandums.
\item Type the SUBJECT of the memorandum on the second line below the last line of an address.
\item Begin the first paragraph of the BODY at the left margin on the third line below the last line of the subject.
\end{enumerate}
\item When used, type the AUTHORITY LINE at the left margin on the second line below the last line of the body.
\item Type the SIGNATURE BLOCK on the fifth line below the authority line or the last line of the body beginning in the center of the page. Identify enclosures, if any; flush with the left margin on the same line as the signature block.
\item Type the OFFICE SYMBOL on the left margin 1 inch from the top edge of the paper. If using an ARIMS record number, space over one space and type the record number in parenthesis.
\item Type the SUBJECT of the memorandum at the left margin on the next line below the office symbol.
\item Begin continuation of the TEXT at the left margin on the third line below the subject.
\begin{enumerate}
\item Do not divide a paragraph of three lines or fewer between pages. At least two lines of the divided paragraph must appear on each page.
\item Include at least two words on each page of any sentence divided between pages.
\item Avoid hyphenation whenever possible.
\item Do not hyphenate a word between pages.
\item Do not type the AUTHORITY LINE (if used) and the SIGNATURE BLOCK on the continuation page without at least two lines of the last paragraph. If, however, the last paragraph or subparagraph has only one line, it may be placed alone on the continuation page with the authority line and signature block.
\end{enumerate}
\item Use the last paragraph to identify the [point of contact], [phone number], and [email] or organization address, as appropriate.
\item Center the page approximately 1 inch from the bottom of the page.
\end{enumerate}
\end{document}
