\documentclass{armymemo}

\address{205th Military Police Battalion}
\address{1234 Charles Road}
\address{Colorado Springs, CO 80904}
\author{James F. Ryan}\rank{PVT}\branch{IN}
\officesymbol{ABC-DEF-GH}
\signaturedate{26 November 2022}
\subject{Army Markdown Tutorial}
\memoline{{MEMORANDUM FOR RECORD}}
\title{Lost Private}
\begin{document}
\begin{enumerate}
 \item This memo is a demonstration of Army markdown, a new way to quickly create memos that will ALWAYS be perfectly formatted IAW with AR 25-50.
\item Don't worry about numbering or lettering. Simply begin each line with a "-". Armymarkdown automatically takes care of everything else.
\item To create an indented section, simply indent the next line 4 spaces (or hit tab in the browser).

\begin{enumerate}
\item This indented section will begin with (a).
\begin{enumerate}
\item We can indent even further.
\begin{enumerate}
\item We can try to indent further, but it will only change from numbers to letters.
\end{enumerate}
\end{enumerate}
\end{enumerate}
\item Now, returning to the original indentation level.
\item Press the "Create Memo PDF" button when you're done!
\item The point of contact for this memorandum is the undersigned available at john.f.ryan@army.mil or (123) 223-3215.
\end{enumerate}
\end{document}
